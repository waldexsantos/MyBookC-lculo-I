  \theoremstyle{plain}          %   bold title, italic body
  \newtheorem{teo}{Teorema}[chapter]
  \newtheorem{lem}{Lema}[chapter]
  \newtheorem{prop}{Proposição}[chapter]
  \newtheorem{corol}{Corolário}[chapter]
  \newtheorem{definition}{Definição}[chapter]
  \theoremstyle{remark}           % italic title, romman body
 % \theoremstyle{definition}       % italic title, romman body
  \newtheorem{obs}{Observação}[chapter]
  \newtheorem{ex}{\bfseries\upshape Exemplo}[chapter]

\newenvironment{dem}{\begin{proof}}{\end{proof}}
 %%%%%%%%%%%%%%%%%%%%%%%%%%%%%%%
%%%% Exercises and Answers %%%%
\usepackage[answerdelayed,lastexercise]{exercise}
\counterwithin{Exercise}{chapter}
\counterwithin{Answer}{chapter}
\renewcommand{\ExerciseHeaderTitle}{({\it \ExerciseTitle})}
\renewcommand{\ExerciseName}{\hspace{-0.6cm}E}
\renewcommand{\ExerciseHeader}{{\textbf{\large\ExerciseName~\ExerciseHeaderNB\ExerciseHeaderTitle\ExerciseHeaderOrigin}}}
\renewcommand{\ExerciseHeader}{\textbf{\ExerciseName\ \ExerciseHeaderNB.}}

% change font for answers header
\renewcommand{\AnswerHeader}{\tiny\textbf{\ExerciseName\ \ExerciseHeaderNB.}\smallskip}
% change font for answers list header
\renewcommand{\AnswerListHeader}{{\tiny\textbf{\AnswerListName\
(\ExerciseListName\ \ExerciseHeaderNB)\ ---\ }}}
%%%%%%%%%%%%%%%%%%%%%%%%%%%%%%


\newenvironment{exer}
{\begin{Exercise}}
{\end{Exercise}}

\newenvironment{resp}
{\begin{Answer}\begin{tiny}}
{\end{tiny}\end{Answer}}

\newenvironment{sol}
 {\begin{proof}[\bfseries\upshape Solução:]}
  {\end{proof}}

%%%%%%%%%%%%%%%%%%%%%%%%%%%%%%
% Exercícios Resolvidos
%%%%%%%%%%%%%%%%%%%%%%%%%%%%%%
\newtheorem{exeresol}{ER}[chapter]
\newenvironment{resol}
{\vspace{-0.5cm}\let\oldqedsymbol=\qedsymbol
  \renewcommand{\qedsymbol}{$\blacksquare$}
  \begin{proof}[\bfseries\upshape Resolução:\hspace{-0.4cm}]~}
  {\end{proof}
  \renewcommand{\qedsymbol}{\oldqedsymbol}}
%%%%%%%%%%%%%%%%%%%%%%%%%%%%%%

\newenvironment{solut}
{\begin{solution}~}
{\end{solution}
\vspace{0.2cm}
\hrule
\vspace{0.2cm}
%\noindent$\rule{0pt}{0pt}$\hspace{-0.8ex}\xrfill[0.7ex]{1 pt}[red]$\blacksquare$\xrfill[0.7ex]{1 pt}[red]
}

%FORMATANDO AS REFERÊNCIAS
\usepackage[%
    alf,
    abnt-emphasize=bf,
    bibjustif,
    recuo=0cm,
    abnt-url-package=url,       % Utiliza o pacote url
    abnt-refinfo=yes,           % Utiliza o estilo bibliográfico abnt-refinfo
    abnt-etal-cite=3,
    abnt-etal-list=3,
    abnt-thesis-year=final
]{abntex2cite}        