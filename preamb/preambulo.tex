\usepackage[brazil]{babel}% escolha de línguas (importado pacote para português e inglês);
\usepackage[utf8]{inputenc}%permite a entrada de caracteres em formato UTF-8 nos arquivos tex;
\usepackage[T1]{fontenc}
\usepackage[a4paper,headheight=15.4pt]{geometry} %conf. página
%________________________________________________________________________________________
% PACOTES MATEMÁTICA
\usepackage{amsmath,amssymb,amsthm} % fontes, símbolos e ambientes adicionais (teorema e derivados)
\everymath{\displaystyle} %simbolos matemático sempre do mesmo tamanho
\usepackage{mathtools}% extensões para facilitar a escrita de fórmulas matemáticas (inclui o pacote amsmath);
\usepackage{cancel} %cancelar termos numa expressão matemática
\usepackage{chngcntr} %possibilita redefinir a numeração dos floats – tabelas, figuras, algoritmos e equações;
\usepackage{array}
%________________________________________________________________________________________
\usepackage{glossaries}%permite a criação de listas de símbolos e abreviaturas;
\usepackage{pdfpages}%faz a inclusão de páginas em PDF no documento final;
\usepackage{verbatim} %insere textos exatamente como digitados
\usepackage{lipsum}%textos aleatorios com o comando \lipsum[n. de parágrafos]
 \usepackage{blindtext}%gerar texto aleatório com os comandos varianes do \blindtext
\usepackage{pifont} %simbolos especiais do pacote pifont
%________________________________________________________________________________________
%LISTAS
\usepackage{paralist}%permite listas especiais
\usepackage{multicol} %MULTICOLUNAS
\usepackage[inline,shortlabels]{enumitem}%listas na mesma linha
%________________________________________________________________________________________
%FIGURAS E TABELAS
\usepackage{float} %usar H em figura
\usepackage{picinpar} %ambiente parpic - figura ao lado do texto
\usepackage{PacotesExtra/picins} %UTILIZAR O COMANDO PARPIC (dependência)
%\usepackage{floatflt} %ambinte floating - fig. ao lado do texto
%\usepackage[table]{xcolor} %colocar cor na primeira linha da tabela
\usepackage{wrapfig} %Necessário para o ambiente wrapfing - figura ao lado do texto contornando
\usepackage{subfigure} %necessário para o ambiente floatingfigure - figura ao lado do texto
\usepackage[pages=some]{background}% colocar IMAGEM DE FUNDO
\usepackage[hypcap=false]{caption}%% configurar o tit. da fig. e tab.-altera a formatação de certas legendas;
\usepackage{rotating} %rotaciona figuras e tabelas com o ambinete \begin{sidewaysfigure} e ambiente landscap
\usepackage{longtable}%tabelas logas que tomam mais de uma página
\usepackage{graphicx,xcolor}%importação e utilização de imagens;
 \usepackage{colortbl} %para tabelas coloridas
 \usepackage{booktabs,tabularx,multirow}	% para tabelas

%\usepackage{subcaption} % % subfigura ou subtabela
%\usepackage{lscape} %Colocar página em formato paisagem
%________________________________________________________________________________________
%FONTES E TEXTO
%\usepackage{lmodern}%carrega a família de fontes Latin Modern, que possui maior abrangência de caracteres;
%\usepackage{times} %carrega fonte Times New Roman;
%\usepackage{uarial}%: carrega fonte URW Arial.
\usepackage{calrsfs} %letras caligráficas
%\usepackage{lettrine} % Para paragrafo no estilo biblia
\usepackage{indentfirst}%%%% indent first line %%%%
\usepackage{icomma}% define a vírgula como separador de decimais no amb. matemático(padrão port. brasileiro);
\usepackage{setspace} %para definir espaçamentos entre linhas 
%________________________________________________________________________________________

\usepackage{pdflscape} % rotaciona uma página na saida pdf
\usepackage{upquote}% copy and paste from PDF (correctly) %
\usepackage{listings}
\usepackage{cite}% citation %
%\usepackage{tabto}
\usepackage{cprotect} %Resolver conflitos do verbatim com o \verb
\usepackage[normalem]{ulem}
%________________________________________________________________________________________
\usepackage{ifthen}% permite a utilização de condições na geração do texto;
%\usepackage{marginnote}% para notas lado da página com o comando \marginnote{texto}[Afast vestical]
\usepackage{fancybox} % para molduras adicionais nas caixas
%PACOTES PARA CITAÇÕES EM CAPÍTULOS (consulta documentação em overleaf)
\usepackage{quotchap}
\usepackage{epigraph}
%==================================
\usepackage{contour} %COLOCAR CONTORNO EM TEXTO
\usepackage{framed}%caixa simples para colocar texto - AMBIENTE FRAMED
\usepackage[listings]{tcolorbox} % caixa de textos personalizadas - ver ARQ. TESTANDO COMANDOS
 \usepackage{varwidth}
 \usepackage{subfiles} % Best loaded last in the preamble (CAP INDEPENDENTES)
%________________________________________________________________________________________
%I##########INDICE REMESSIVO#################################################
\usepackage{imakeidx} %permite a criação de um índice remissivo ao fim do texto;
%\makeindex[options= -s IndexStyle.ist]
\makeindex[columns=3, title=Índice Remissivo,   options= -s IndexStyle.ist]
%________________________________________________________________________________________
%########CONFIG. PDF LINKS##############################################
\usepackage[hyphens,spaces,obeyspaces]{url} %gera os links entre referências no PDF;
%\usepackage{varioref} %utilizar o comando \vref que imprime algo tipo "5 on page 25"

\usepackage[pdfborder={0 0 0 [0 0]},colorlinks=true,linktocpage=true,linkcolor=red,citecolor=cyan,filecolor=blue,urlcolor=blue,pdfauthor={Valdex Santos},pdftitle={Cálculo de uma Variável Real - Vol. I}]{hyperref}

%cCONFIGURAÇÕES DE CABEÇALHO
\usepackage{fancyhdr}
\pagestyle{fancy}
\fancyhf{}
\fancyhead[RE]{Cálculo Diferencial e Integral}
\fancyhead[LO]{\rightmark}
\fancyhead[LE,RO]{\thepage}
%license footnote
\cfoot{\tiny{\copyright Valdex Santos}}
%*************************************************************
%%% OUTRAS CONFIGURAÇÕES %%%%
\let\cleardoublepage\clearpage% no blank pages between chapters
\DeclareTextFontCommand{\emph}{\bfseries}%emphasis \emph
\def\E#1{\mathrm{E}\!#1\!}
\def\NaN{\mathrm{NaN}\!}

%&&&&&&&&ENTRADA DE ARQUIVOS &&&&&&&&&&&&&&&&&&&&&&&&
\newcommand{\sen}{\operatorname{sen}\,}
\newcommand{\senh}{\operatorname{senh}\,}
\renewcommand{\sin}{\operatorname{sen}\,}
\renewcommand{\sinh}{\operatorname{senh}\,}
\newcommand{\tg}{\operatorname{tg}\,}
\newcommand{\arc}{\operatorname{arc}}
%\newcommand{\tg}{\operatorname{tg}}
\newcommand{\cotg}{\operatorname{cotg}}
\newcommand{\cosec}{\operatorname{cosec}}
\newcommand{\cossec}{\operatorname{cossec}}
\newcommand{\p}{\partial}
\newcommand{\dd}{\mathrm{d}}
\newcommand{\Dom}{\operatorname{Dom}}
\newcommand{\diag}{\operatorname{diag}}
\newcommand{\proj}{\operatorname{proj}}
\newcommand{\dist}{\operatorname{dist}}
\newcommand{\sign}{\operatorname{sign}}
\newcommand{\am}{$a$}
\newcommand{\f}{$f$}
\newcommand{\g}{$g$}
\newcommand{\hm}{$h$}
\newcommand{\x}{$x$}
\newcommand{\y}{$y$}
\newcommand{\xo}{$x_0$}

%COMANDO DE LINK GEOGEBRA
\newcommand{\geogebra}{\href{https://www.geogebra.org/}{GeoGebra}}
%%%% indexing %%%%

%&&&&&&&&&&&&&&&&&&&&&&&&&&&&&&&&&&&&&&&&&&&&&&&&&&&&&&&&&&&&&&&&&&&&&&&&&&&&&&&&&&&&&
%COMANDOS DE AMBIENTES MATEMÁTICO
\newcommand{\mat}[1]{$#1$}%modo matemático%
  \newcommand{\matt}[1]{$$#1$$}%mdoo matemático em destaque
  \newcommand{\sistema}[2]{$$\left\{\begin{array}{l} #1\\\\ #2\end{array}\right.$$}
\newcommand{\sistemaeq}[3][1]{\begin{equation}\left\{\begin{array}{l} #2\\\\ #3\end{array}\right.\label{#1}\end{equation}}
\newcommand{\intg}[1]{\int #1 \, dx} %Integral indefinida
\newcommand{\intdef}[3]{\int_{#1}^{#2} #3 \, dx} %integral definida
%%%%% COMANDOS MATEMÁTICOS DE MATRIZ E VETORES  %%%%%%%%%%%%%
\newcommand{\matdd}[4]{\begin{bmatrix} #1&#2\\#3&#4 \end{bmatrix}}
\newcommand{\matddd}[9]{\begin{bmatrix} #1&#2&#3 \\ #4&#5&#6 \\ #7&#8&#9 \end{bmatrix}}
\newcommand{\vetdd}[2]{\begin{bmatrix} #1 \\#2 \end{bmatrix}}
\newcommand{\vetddd}[3]{\begin{bmatrix} #1 \\ #2\\ #3 \end{bmatrix}}
\newcommand{\dica}{\ding{43}}
%&&&&&&&&&&&&&&&&&&&&&&&&&&&&&&&&&&&&&&&&&&&&&&&&&&&&&&&&&&&&&&&&&&&&&&&&&&&&&&&&&&&&&&&&

%NOVOS COMANDOS PARA PARÊNTESES, COLCHETES E CHAVES
\newcommand{\pr}[1]{\ensuremath{\left[#1\right]}} %parênsteses retos
\newcommand{\pc}[1]{\ensuremath{\left(#1\right)}} %oparenteses curvos
\newcommand{\barra}[1]{\ensuremath{\left|#1\right|}}%barra
\newcommand{\chav}[1]{\ensuremath{\left\{#1\right\}}}%chaves
%######################################################################


%CORES#########################################
\newcommand{\red}[1]{{\color{red}{#1}}}
\newcommand{\blu}[1]{{\color{blue}{#1}}}
\newcommand{\gre}[1]{{\color{\green}{#1}}}
%========================================

%DESAFIO E ATENÇÃO
\newcommand{\desafio}[1]{{\contour{green}{\color{black}\scshape Desafio:} \textsl{#1}}}
\newcommand{\atencao}[1]{{\contour{red}{\color{black}\scshape Atenção:} \textsl{#1}}}
\newcommand\Ccancel[2][black]{\renewcommand\CancelColor{\color{#1}}\cancel{#2}}%MUDANÇA DE COR AO CANCLER TEXTO

%COMANDO SOLUÇÃO
\newcommand{\solution}{\noindent\textbf{Solução:}}




%NOVO AMBIENTE MINIPAGE
\newcommand{\minipag}[2]{\begin{minipage}{#1\textwidth} #2\end{minipage}} %ambiente minipage


%LINHA NO FINAL DA SOLUÇÃO DO EXEMPLO
\newcommand{\fimex}{\xrfill[0.7ex]{1 pt}[red]$\blacksquare$\xrfill[0.7ex]{1 pt}[red]}

%TEXTO EM TORNO DA FIGURA-RENOMEANDO AMBIENTE PARPIC-MINIPAGE
\newcommand{\figtext}[4]{%
\pichskip{#1}%distância entre texto e figura
\parpic[#2]{%localização direita ou esquerda (l or r)
\begin{minipage}{#3\textwidth}
#4
\end{minipage}
}%
}%

%NEGRITO NO MODO MATEMÁTICO E INSERÇÃO DE FÓRMULAS NOS TÍTULOS
\newcommand{\boldmat}[1]{\boldmath #1 \unboldmath}
\newcommand{\formula}[1]{\texorpdfstring{#1}{}}
%=========================================================
%COMANDO DE NOTA
\newcommand{\nota}[1]{%
\vspace{0.4cm}
\hrule
\hspace{-0.6cm}\begin{minipage}{0.1\textwidth}
 \noindent\includegraphics{preamb/figs/FigNota.jpg}
 \end{minipage}\hfill
 \begin{minipage}{0.85\textwidth}
\vspace{0.2cm}
\hspace{-0.4cm}\textbf{Nota}
\vspace{0.2cm}

  #1
  \end{minipage}
  \vspace{0.2cm}
\hrule \vspace{0.3cm}
}
\newcommand{\emconstrucao}{
  \begin{tabular}{|c|}\hline
    Em construção ... Gostaria de participar na escrita deste livro?\\ Procure o Prof.  \href{mailto:waldexsantos@gmail.com}{Valdex Santos}\\\hline
  \end{tabular}
}

\newcommand{\construirSec}{
\begin{tabular}{|c|}\hline
  Esta seção (ou subseção) está sugerida. Participe da sua escrita.\\ Procure o prof.  \href{mailto:waldexsantos@gmail.com}{Valdex Santos}\\\hline
\end{tabular}
}

\newcommand{\construirExeresol}{
  \begin{center}
    \begin{tabular}{|c|}\hline
      Esta seção carece de exercícios resolvidos. Participe da sua escrita.\\
      Procure o prof.  \href{mailto:waldexsantos@gmail.com}{Valdex Santos}\\\hline
    \end{tabular}
  \end{center}
}

\newcommand{\construirExer}{
  \begin{center}
    \begin{tabular}{|c|}\hline
      Esta seção carece de exercícios. Participe da sua escrita.\\
     Procure o Prof. \href{mailto:waldexsantos@gmail.com}{Valdex Santos}\\
     \hline
    \end{tabular}
  \end{center}
}

\newcommand{\construirResp}{
\begin{center}
  Este exercício está sem resposta sugerida. Proponha uma resposta.\\
  \Procure o Prof. \href{mailto:waldexsantos@gmail.com}{Valdex Santos}
\end{center}
}



%=========================================================================================
%DEFININDO NOVO ESTILO DE CAPÍTULO E SEÇÃO- VER PAG.: https://www.overleaf.com/learn/latex/Sections_and_chapters#Document_chapters_and_sections_in_a_Book%2FReport
\usepackage{titlesec}
 \titleformat
{\chapter} % command
[display] % shape
{\bfseries\LARGE\itshape} % format
{Capítulo  \ \thechapter} % label
{0.8ex} % sep
{
    \rule{\textwidth}{1pt}
    \vspace{0.5ex}
    \centering
} % before-code
[
\vspace{-0.5ex}%
\rule{\textwidth}{0.3pt}
] % after-code
 
 %SEÇÃO
\titleformat{\section}[hang]
{\Large\bfseries}
{\thesection.}{0.5em}{}
 
\titlespacing*{\section}{0pt}{*2}{*2}
%==========================================================================================
% DEFININDO NOVO AMBIENTE TEXTO AO LADO
\newenvironment{Sidebar}
   {\wrapfigure{r}{0.43\textwidth}\vspace{-3em}\tcolorbox[colframe=black!50!white]\footnotesize}
   {\endtcolorbox\vspace{2em}\endwrapfigure}
   \begin{comment}
   \begin{Sidebar}
{\centering\bfseries Title \\[0.5em]}
\blindtext[1]
\end{Sidebar} 
\end{comment}
  
  
%MBINENTE PARA RECUO COM QUOTE E COMPACTENUM PRA OPOÇÕES NOS EXERCICIOS
\newenvironment{opcexer}[1]
{%
    \begin{quote}
    \vspace{-0.2cm}
    \onehalfspacing
       \begin{enumerate}[#1]
}%
 {%
    \end{enumerate}
    \end{quote}
}%





%%PARA O AMBIENTE TIKZ (VER ARQ. TESTANDO COMANDOS)
\usepackage{tikz} % pacote gráfico
\usetikzlibrary{babel} % para compatiblidade com o pacote babel, requerido por agumas bibliotecas como o cd.
\usetikzlibrary{calc} % calc eh para efetuar calculos matematicos ou expressoes em coordenadas
\usetikzlibrary{through} % circulo passando por ponto, por exemplo.
\usetikzlibrary{patterns} % preenchimentos
\usetikzlibrary{intersections} % interseccao entre caminhos
\usetikzlibrary{matrix} % matriz no tikz
\usetikzlibrary{cd} % diagrama comutativa
\usetikzlibrary{arrows.meta, intersections}
%&&&&&&&&&&&&&&&&&&&&&&&&&&&&&&&&&&&&&&&&&&&&&&&&&&&&
\setcounter{secnumdepth}{5}
\setcounter{tocdepth}{5}


\usepackage{xhfill}
\newcommand{\xfill}[2][1ex]{{%
  \dimen0=#2\advance\dimen0 by #1
  \leaders\hrule height \dimen0 depth -#1\hfill%
}}
\newcommand{\xfilll}[2][1ex]{%
  \dimen0=#2\advance\dimen0 by #1%
  \leaders\hrule height \dimen0 depth -#1\hfill%
}
%#########################################################
%PACOTE MINITOC PARA SUBMENUS NOS CAPÍTULOS com o comando minitoc
\usepackage{minitoc}
\renewcommand{\mtctitle}{Conteúdos do Capítulo} %Renomeando os subcapítulos
%####################################################
\usepackage{bookmark} % to add bookmark manually
 \renewcommand{\arraystretch}{1.5} %space between rows in tables



\input preamb/preambulo_book.tex



