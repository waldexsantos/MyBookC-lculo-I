\newcommand{\sen}{\operatorname{sen}\,}
\newcommand{\senh}{\operatorname{senh}\,}
\renewcommand{\sin}{\operatorname{sen}\,}
\renewcommand{\sinh}{\operatorname{senh}\,}
\newcommand{\tg}{\operatorname{tg}\,}
\newcommand{\arc}{\operatorname{arc}}
%\newcommand{\tg}{\operatorname{tg}}
\newcommand{\cotg}{\operatorname{cotg}}
\newcommand{\cosec}{\operatorname{cosec}}
\newcommand{\cossec}{\operatorname{cossec}}
\newcommand{\p}{\partial}
\newcommand{\dd}{\mathrm{d}}
\newcommand{\Dom}{\operatorname{Dom}}
\newcommand{\diag}{\operatorname{diag}}
\newcommand{\proj}{\operatorname{proj}}
\newcommand{\dist}{\operatorname{dist}}
\newcommand{\sign}{\operatorname{sign}}
\newcommand{\am}{$a$}
\newcommand{\f}{$f$}
\newcommand{\g}{$g$}
\newcommand{\hm}{$h$}
\newcommand{\x}{$x$}
\newcommand{\y}{$y$}
\newcommand{\xo}{$x_0$}

%COMANDO DE LINK GEOGEBRA
\newcommand{\geogebra}{\href{https://www.geogebra.org/}{GeoGebra}}
%%%% indexing %%%%

%&&&&&&&&&&&&&&&&&&&&&&&&&&&&&&&&&&&&&&&&&&&&&&&&&&&&&&&&&&&&&&&&&&&&&&&&&&&&&&&&&&&&&
%COMANDOS DE AMBIENTES MATEMÁTICO
\newcommand{\mat}[1]{$#1$}%modo matemático%
  \newcommand{\matt}[1]{$$#1$$}%mdoo matemático em destaque
  \newcommand{\sistema}[2]{$$\left\{\begin{array}{l} #1\\\\ #2\end{array}\right.$$}
\newcommand{\sistemaeq}[3][1]{\begin{equation}\left\{\begin{array}{l} #2\\\\ #3\end{array}\right.\label{#1}\end{equation}}
\newcommand{\intg}[1]{\int #1 \, dx} %Integral indefinida
\newcommand{\intdef}[3]{\int_{#1}^{#2} #3 \, dx} %integral definida
%%%%% COMANDOS MATEMÁTICOS DE MATRIZ E VETORES  %%%%%%%%%%%%%
\newcommand{\matdd}[4]{\begin{bmatrix} #1&#2\\#3&#4 \end{bmatrix}}
\newcommand{\matddd}[9]{\begin{bmatrix} #1&#2&#3 \\ #4&#5&#6 \\ #7&#8&#9 \end{bmatrix}}
\newcommand{\vetdd}[2]{\begin{bmatrix} #1 \\#2 \end{bmatrix}}
\newcommand{\vetddd}[3]{\begin{bmatrix} #1 \\ #2\\ #3 \end{bmatrix}}
\newcommand{\dica}{\ding{43}}
%&&&&&&&&&&&&&&&&&&&&&&&&&&&&&&&&&&&&&&&&&&&&&&&&&&&&&&&&&&&&&&&&&&&&&&&&&&&&&&&&&&&&&&&&

%NOVOS COMANDOS PARA PARÊNTESES, COLCHETES E CHAVES
\newcommand{\pr}[1]{\ensuremath{\left[#1\right]}} %parênsteses retos
\newcommand{\pc}[1]{\ensuremath{\left(#1\right)}} %oparenteses curvos
\newcommand{\barra}[1]{\ensuremath{\left|#1\right|}}%barra
\newcommand{\chav}[1]{\ensuremath{\left\{#1\right\}}}%chaves
%######################################################################


%CORES#########################################
\newcommand{\red}[1]{{\color{red}{#1}}}
\newcommand{\blu}[1]{{\color{blue}{#1}}}
\newcommand{\gre}[1]{{\color{\green}{#1}}}
%========================================

%DESAFIO E ATENÇÃO
\newcommand{\desafio}[1]{{\contour{green}{\color{black}\scshape Desafio:} \textsl{#1}}}
\newcommand{\atencao}[1]{{\contour{red}{\color{black}\scshape Atenção:} \textsl{#1}}}
\newcommand\Ccancel[2][black]{\renewcommand\CancelColor{\color{#1}}\cancel{#2}}%MUDANÇA DE COR AO CANCLER TEXTO

%COMANDO SOLUÇÃO
\newcommand{\solution}{\noindent\textbf{Solução:}}




%NOVO AMBIENTE MINIPAGE
\newcommand{\minipag}[2]{\begin{minipage}{#1\textwidth} #2\end{minipage}} %ambiente minipage


%LINHA NO FINAL DA SOLUÇÃO DO EXEMPLO
\newcommand{\fimex}{\xrfill[0.7ex]{1 pt}[red]$\blacksquare$\xrfill[0.7ex]{1 pt}[red]}

%TEXTO EM TORNO DA FIGURA-RENOMEANDO AMBIENTE PARPIC-MINIPAGE
\newcommand{\figtext}[4]{%
\pichskip{#1}%distância entre texto e figura
\parpic[#2]{%localização direita ou esquerda (l or r)
\begin{minipage}{#3\textwidth}
#4
\end{minipage}
}%
}%

%NEGRITO NO MODO MATEMÁTICO E INSERÇÃO DE FÓRMULAS NOS TÍTULOS
\newcommand{\boldmat}[1]{\boldmath #1 \unboldmath}
\newcommand{\formula}[1]{\texorpdfstring{#1}{}}
%=========================================================
%COMANDO DE NOTA
\newcommand{\nota}[1]{%
\vspace{0.4cm}
\hrule
\hspace{-0.6cm}\begin{minipage}{0.1\textwidth}
 \noindent\includegraphics{preamb/figs/FigNota.jpg}
 \end{minipage}\hfill
 \begin{minipage}{0.85\textwidth}
\vspace{0.2cm}
\hspace{-0.4cm}\textbf{Nota}
\vspace{0.2cm}

  #1
  \end{minipage}
  \vspace{0.2cm}
\hrule \vspace{0.3cm}
}