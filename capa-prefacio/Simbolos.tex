\begin{table}[]
    \centering
    \begin{tabular}{l|l}
       Símbolo & Significado & Símbolo & Significado
  
\(\in\) & pertence &\(\notin\)&não pertence\\
\(\subseteq\) &  está contido, podendo ser igual &  \(\subset\) &  está contido propriamente, não podendo ser igual
  \(\not\subset\) &  não está contido propriamente, nem é igual &  \(\supseteq\) &    contém, podendo ser igual\\
 \(\supset\)&  contém propriamente, não podendo ser igual  &\(\not\supset\) &  não contém propriamente, nem é igual\\
  \(\emptyset\) &  conjunto vazio&  \(\vert\) &  tal que\\
  \(\forall\) &  para todos (ou qualquer que seja)&  \(\mathbb{N}\) &  conjunto dos números naturais\\
&   \(\mathbb{Q}\) &  conjunto dos números racionais &  \(\mathbb{Z}\) &  conjunto dos números inteiros\\
 \(\mathbb{R}\) &  conjunto dos números reais &  \( \exists \)&  existe\\
 \( \nexists \) &  não existe &  \( \forall \) &  para todo (ou qualquer que seja)\\
 \( \rightarrow \) &  implica logicamente que &  \( \leftrightarrow \) &  equivale logicamente a; se e somente se lógico, implica bidirecional e logicamente que\\
 \( \Rightarrow \) &  implica materialmente que, se então material &  \( \Leftrightarrow \) &  equivale materialmente a; se e somente se material, implica bidirecional e materialmente que
 
\end{tabular}
    \caption{Caption}
    \label{tab:my_label}
\end{table}