%Este trabalho está licenciado sob a Licença Creative Commons Atribuição-CompartilhaIgual 3.0 Não Adaptada. Para ver uma cópia desta licença, visite https://creativecommons.org/licenses/by-sa/3.0/ ou envie uma carta para Creative Commons, PO Box 1866, Mountain View, CA 94042, USA.
\cleardoublepage
\chapter*{Prefácio}
\addcontentsline{toc}{chapter}{Prefácio}

Este book tem como objetivo apresentar, de maneira concisa, os conceitos e resultados do
Cálculo Diferencial e Integral de uma variável e está dividido em seis capítulos que tratam especificamente dos seguintes assuntos: Fundamentos sobre Funções; Limites e Continuidade; Derivadas e Aplicações, Integrais e Aplicações.

O texto está escrito em uma linguagem precisa e esclarecedora. Precisa, porque a Matemática não
pode ser construída sem o devido rigor na linguagem e na lógica de suas proposições; esclarecedora, porque desejamos evitar o aparecimento de definições e nomenclaturas desnecessárias, que dificultem o caminhar do estudante durante a leitura desta obra.

Neste sentido, preferimos não apresentar as demonstrações de  Teoremas, proposições e colorários que podem ser encontrados em livros de Cálculo I, tais como: \citeonline{IezziLimDerInt}, \citeonline{LeitholdVol1}, \citeonline{StewartSiagleVariable}, dentre outros. É importante acessar a página \href{waldexifba.wordpress.com}{waldexifba.wordpress.com} para acesso a outros temas do Cálculo diferencial e integral.

Durante o Curso utilizaremos o software \geogebra~como metologia auxiliar na aprendizagem dos conteúdos. O GeoGebra é um software de Geometria Dinâmica amplamente utilizado no mundo e que permite trabalhar ao mesmo tempo com a parte algébrica e geométrica dos objetos matemáticos, além de permitir explorar conteúdos através de planilhas. Com ele, as manipulações algébricas são refletidas automaticamente em suas representações geométricas e vice versa.

Lembramos que por ser um texto a ser aplicado em disciplinas de cálculo no ano letivo de 2020, estará  em processo de construção não contendo de imediato todas as informações propostas, mas que no decorrer do ano isso deverá estar concretizado. Neste sentido, contamos com a colaboração dos colegas e estudantes, nos auxiliando a escrever texto, exercícios (propostos e resolvidos), aplicações e sinalizando correções a serem feitas. Sugestões são sempre bem vindas.

\textbf{Bom estudo a todos!}